\documentclass[12pt]{article}
\usepackage{geometry}
\usepackage{graphicx}
\usepackage{amsmath}
\usepackage{enumitem}
\usepackage{parskip}
\usepackage{setspace}
\usepackage{float}
\usepackage{relsize}
\usepackage{listings}
\usepackage{fancyvrb}
\usepackage{hyperref}

\onehalfspacing
\geometry{letterpaper, portrait, margin=1in}

\hypersetup{
    colorlinks=true, 
    urlcolor=blue
}

\begin{document}
\noindent CSCI 4962 \hfill Project Proposal \\
Siddha Kilaru \\

\begin{description}
    \item[Project Title] \hfill \\
    Using Neural Networks as Numeric Solvers.

    \item[Team Members] \hfill \\
    Siddha Kilaru

    \item[Summary] \hfill \\
    The goal of this project is to read the paper
    \url{https://arxiv.org/pdf/1711.10561.pdf} (Physics Informed Neural
    Networks) and understand the material. After doing so, I plan to
    implement/recreate what the paper did; it seems they used neural networks
    to help solve (numerically) two different differential equations (Burgers'
    and Shrodinger Equation). I also plan on solving other simpler differential
    equations that model a spring damper system, pendulum, etc., and comparing
    them to traditional numerical solvers that are known to work well on them.
    I am not sure how much I will be able to accomplish due to foreseeable
    obstacles, but the main things are the goals mentioned above.

    \item[Machine Learning Use] \hfill \\
    Machine learning is involved because it is used to approximate a function,
    and the function, in this case, is a solution to a differential equation.
\end{description}



\end{document}