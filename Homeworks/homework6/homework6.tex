\documentclass[12pt]{article}
\usepackage{geometry}
\usepackage{graphicx}
\usepackage{amsmath}
\usepackage{enumitem}
\usepackage{parskip}
\usepackage{setspace}
\usepackage{float}
\usepackage{relsize}
\usepackage{listings}
\usepackage{fancyvrb}
\usepackage{hyperref}
\hypersetup {
    colorlinks=true,
    linkcolor=blue,
    filecolor=magenta,      
    urlcolor=cyan,
    pdftitle={Overleaf Example},
    pdfpagemode=FullScreen,
}

\onehalfspacing
\geometry{letterpaper, portrait, margin=1in}


\begin{document}
\noindent CSCI 4962 \hfill Homework 6 \\
Siddha Kilaru \\

\begin{description}
    \item[Task 1] \hfill \\
    Generally, reinforcement learning problems have three main factors, the
    agent, policy, and environment. The environment is essentially the specific
    game the agent is playing, such as tic-tac-toe or a robot navigating a
    course. At any specific time, the agent is in a specific state in the
    environment, and the agent refers to its policy to decide what to do. In
    other words, the policy is a function that maps states to actions. One way
    to mathematically formalize and model this problem is to use a Markov
    decision process (MDP). An MDP can be fully characterized by just five
    things, a set of states ($S$), a set of actions ($A$), a distribution
    containing the probabilities that taking $a\in A$ in $s\in S$ will lead to
    some $s'\in S$, a reward function for transitioning from $s \in S$ to $s'
    \in S$, and lastly a discount factor $\gamma$ that decays the rewards over
    time. 

    Let us consider a reinforcement learning problem where a rover (robot) with
    a limited battery life must collect as many eggs as possible. The eggs are
    randomly distributed around the course. If the rover's battery is running
    low, the rover can return to the home position and recharge. The game is
    over if the rover runs out of battery in the middle of the course. An MDP
    can model this game. First, the set of states is a battery level greater
    than 20\% and a battery level less than or equal to 20\%. The set of
    actions is either search for eggs or hold the current position. The reward
    function is proportional to the number of eggs collected. A high-level
    overview of the transition model is as follows: if the battery is at a high
    state, the rover can either search or hold position, and if the battery is
    at a low state, the rover can either search or hold the position. From
    this, one can immediately see that if the rover is in a low battery state,
    it is optimal to hold the position and recharge. Conversely, if the rover
    is in a high battery state, it is optimal to search. A more complex optimal
    decision that the rover could make is when it starts to see its battery run
    low; it starts to search in the direction of the home position, where it
    can recharge. The rover would learn this over many iterations with the help
    of the reward function and the discount factor. 

    \item[Task 2] \hfill \\
    Reinforcement Learning is highly applicable and related to healthcare
    problems. The process that doctors in health care undergo is very similar
    to the kind of decision-making process an agent makes in a reinforcement
    learning problem. For instance, a doctor would diagnose and suggest
    specific treatments to a patient, and if the patient's health improves, the
    doctor continues to offer the current treatment. If the patient's health
    worsens, perhaps there was a misdiagnosis, or other treatments are better.
    This sequential decision process can be modeled by a reinforcement learning
    problem. More specifically, an MDP can model this kind of process.

    Consider this open-source project
    \url{https://github.com/microsoft/med-deadend}.This project is a
    reinforcement learning model that identifies certain treatments doctors
    should avoid prescribing because they will lead to medical dead ends. A
    medical dead end is a state where all further actions/or in-action would
    lead to a patient's death. The transition model is fairly simple.
    Essentially, the state space consists of all the possible medical
    conditions a person has. The action space consists of all the possible
    prescriptions the doctor can make to a patient. The basic idea for the most
    optimal transition model is as follows: if a patient is at state $s$ and
    action $a$ causes the state to transition to a medical dead end with
    probability $p$, then the policy picks $a$ at $s$ with probability $1-p$.
    More formally, this is described by the optimal value functions $Q_D^*(s,
    a)$ and $Q_R^*(s, a)$. The project above aims to learn these functions and
    then can be used to identify medical dead ends in practice.

    The model was tested on sepsis, and the researchers obtained exciting
    results. First, the results indicated ``that more than 12 percent of
    treatments given to non-surviving patients could be detrimental 24 hours
    before death.''[1] Also, they identified that ``2.7 percent of
    non-surviving patients entered medical dead-end trajectories with a sharply
    increasing rate up to 48 hours before death, and close to 10 percent when
    we slightly relaxed our thresholds for predicting medical dead-ends.''[1]
    These results can significantly impact the real world, potentially saving
    thousands of lives.

    [1] Mehdi Fatemi, Taylor W. Killian, Jayakumar Subramanian, Marzyeh
    Ghassemi: “Medical Dead-ends and Learning to Identify High-risk States and
    Treatments”, 2021.
\end{description}



\end{document}