\documentclass[12pt]{article}
\usepackage{geometry}
\usepackage{graphicx}
\usepackage{amsmath}
\usepackage{enumitem}
\usepackage{parskip}
\usepackage{setspace}
\usepackage{float}
\usepackage{relsize}
\usepackage{listings}
\usepackage{fancyvrb}

\onehalfspacing
\geometry{letterpaper, portrait, margin=1in}


\begin{document}
\noindent CSCI 4962 \hfill Discussion 7 \\
Siddha Kilaru \\

\begin{description}
    \item[Paper 1] \hfill \\
    ``Finite-time Analysis of the Multiarmed Bandit Problem'' by Peter Auer,
    Nicolo Cesa-Bianchi, and Paul Fischer:

    The $K$-armed bandit problem is given a set of $K$ random variables that
    are independent and not necessarily independent from one another, which is
    the best one to sample from over time to maximize utility. An algorithm
    called a policy chooses the following random variable to sample from given
    data from prior sampling. This paper is about empirically and analytically
    analyzing how the regret of certain policies changes over time. Moreover,
    the paper uses a proof that says the regret follows a logarithmic curve
    over time, and the research paper was able to recreate this using different
    policies such as $UCB$1, $UCB$2, and $\epsilon$-greedy. One way to extend
    the paper is to drop the stationarity assumption of the $K$-armed bandit
    problem, which turns it into a stochastic $K$-armed bandit problem. Here,
    there needs to be more analysis done on how regret behaves over time
    because the problem is much more general. This analysis can be purely
    analytical, or maybe some conclusions can be drawn from using an algorithm
    (policy) designed with the stationarity assumption in mind.
    \item[Paper 2]
    \item[Paper 3]

    

\end{description}




\end{document}