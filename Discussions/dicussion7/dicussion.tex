\documentclass[12pt]{article}
\usepackage{geometry}
\usepackage{graphicx}
\usepackage{amsmath}
\usepackage{enumitem}
\usepackage{parskip}
\usepackage{setspace}
\usepackage{float}
\usepackage{relsize}
\usepackage{listings}
\usepackage{fancyvrb}

\onehalfspacing
\geometry{letterpaper, portrait, margin=1in}


\begin{document}
\noindent CSCI 4962 \hfill Discussion 7 \\
Siddha Kilaru \\
    
% ``Finite-time Analysis of the Multiarmed Bandit Problem'' by Peter Auer,
%   Nicolo Cesa-Bianchi, and Paul Fischer:

\begin{description}
    \item[Paper 1] \hfill \\
    Auer, Cesa-Bianchi, Fischer, ``Finite-time Analysis of the Multiarmed Bandit
    Problem,'' MLJ, 2002.

    The $K$-armed bandit problem is given a set of $K$ random variables that
    are independent and not necessarily independent from one another, which is
    the best one to sample from over time to maximize utility. An algorithm
    called a policy chooses the following random variable to sample from given
    data from prior sampling. This paper is about empirically and analytically
    analyzing how the regret of certain policies changes over time. Moreover,
    the paper uses a proof that says the regret follows a logarithmic curve
    over time, and the research paper was able to recreate this using different
    policies such as $UCB$1, $UCB$2, and $\epsilon$-greedy. One way to extend
    the paper is to drop the stationarity assumption of the $K$-armed bandit
    problem, which turns it into a stochastic $K$-armed bandit problem. Here,
    there needs to be more analysis done on how regret behaves over time
    because the problem is much more general. This analysis can be purely
    analytical, or maybe some conclusions can be drawn from using an algorithm
    (policy) designed with the stationarity assumption in mind.
    \item[Paper 2] \hfill \\
    Mnih, Kavukcuoglu, Silver, Rusu, Veness, et al., ``Human Level Control
    Through Deep Reinforcement Learning,'' Nature, 2015.

    This paper is concerned with introducing deep learning methods to
    reinforcement learning. The paper focuses on training an agent to play
    classic Atari 2600 games. The paper mentions that traditional reinforcement
    learning methods like Q-learning have been used on these games and yielded
    promising results; however, combing Q-learning with deep neural networks to
    effectively create Deep Q-networks (DQN) performed exponentially better
    than the linear learning model. The design of the DQN used consisted of
    convolutions to a dense layer (very similar to a CNN), where the inputs
    were pixel data, and the outputs were controller actions. The Q-learning
    aspect comes into play when the weights need to be updated. Unlike
    traditional deep learning, there are extra steps involved in the
    optimization process. The results are shown in figure 3 of the paper. We
    can see that most of the games played by a DQN outperformed the best linear
    learner. Moreover, the paper mentions that the DQN was comparable in
    performance to a professional human player. One way to extend this research
    is to study the trained DQNs and extract strategies from them. For example,
    we can correlate a general game state with the next most optimal play. This
    kind of insight can be used by human players looking to improve their
    scores in-game or learn patterns that were not obvious to them.


    \item[Paper 3]

    

\end{description}




\end{document}